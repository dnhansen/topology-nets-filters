% Document setup
\documentclass[article, a4paper, 11pt, oneside]{memoir}
\usepackage[utf8]{inputenc}
\usepackage[T1]{fontenc}
\usepackage[UKenglish]{babel}

% Document info
\newcommand\doctitle{Notes on nets and filters}
\newcommand\docauthor{Danny Nygård Hansen}

% Formatting and layout
\usepackage[autostyle]{csquotes}
\usepackage[final]{microtype}
\usepackage{xcolor}
\frenchspacing
\usepackage{latex-sty/articlepagestyle}
\usepackage{latex-sty/articlesectionstyle}
% \usepackage{latex-sty/amalgsymbol}

% Fonts
\usepackage[largesmallcaps]{kpfonts}
\DeclareSymbolFontAlphabet{\mathrm}{operators} % https://tex.stackexchange.com/questions/40874/kpfonts-siunitx-and-math-alphabets
\linespread{1.06}
\let\mathfrak\undefined
\usepackage{eufrak}
\usepackage{inconsolata}
% \usepackage{amssymb}

% Hyperlinks
\usepackage{hyperref}
\definecolor{linkcolor}{HTML}{4f4fa3}
\hypersetup{%
	pdftitle=\doctitle,
	pdfauthor=\docauthor,
	colorlinks,
	linkcolor=linkcolor,
	citecolor=linkcolor,
	urlcolor=linkcolor,
	bookmarksnumbered=true
}

% Equation numbering
\numberwithin{equation}{chapter}

% Footnotes
\footmarkstyle{\textsuperscript{#1}\hspace{0.25em}}

% Mathematics
\usepackage{latex-sty/basicmathcommands}
\usepackage{latex-sty/framedtheorems}
\usepackage{latex-sty/topologycommands}
\usepackage{tikz-cd}
\tikzcdset{arrow style=math font} % https://tex.stackexchange.com/questions/300352/equalities-look-broken-with-tikz-cd-and-math-font
\usetikzlibrary{babel}

% Lists
\usepackage{enumitem}
\setenumerate[0]{label=\normalfont(\alph*)}
\setlist{  
  listparindent=\parindent,
  parsep=0pt,
}

% Bibliography
\usepackage[backend=biber, style=authoryear, maxcitenames=2, useprefix]{biblatex}
\addbibresource{references.bib}

% Title
\title{\doctitle}
\author{\docauthor}

\newcommand{\setF}{\mathbb{F}}
\newcommand{\ev}{\mathrm{ev}}
\newcommand{\calT}{\mathcal{T}}
\newcommand{\calU}{\mathcal{U}}
\newcommand{\calB}{\mathcal{B}}
\newcommand{\calE}{\mathcal{E}}
\newcommand{\calC}{\mathcal{C}}
\newcommand{\calD}{\mathcal{D}}
\newcommand{\calF}{\mathcal{F}}
\newcommand{\calG}{\mathcal{G}}
\newcommand{\calM}{\mathcal{M}}
\newcommand{\calA}{\mathcal{A}}
\newcommand{\calP}{\mathcal{P}}
\newcommand{\calR}{\mathcal{R}}
\newcommand{\calO}{\mathcal{O}}
\newcommand{\strucS}{\mathfrak{S}}
\DeclarePairedDelimiter{\gen}{\langle}{\rangle} % Generating set
\newcommand{\frakL}{\mathfrak{L}}
\newcommand{\frakN}{\mathfrak{N}}
\newcommand{\frakA}{\mathfrak{A}}
\newcommand{\frakB}{\mathfrak{B}}
\newcommand{\ab}{\mathit{ab}}

\DeclareMathOperator{\im}{im}
\DeclareMathOperator{\coker}{coker}
\DeclareMathOperator{\stab}{Stab}

% Categories
\newcommand{\cat}[1]{\mathcal{#1}}
\newcommand{\scat}[1]{\mathbf{#1}} % category supposed to be small
\newcommand{\ncat}[1]{\mathbf{#1}} % named categories like Set, Top

\newcommand{\catSet}{\ncat{Set}} % Category of sets
\newcommand{\catGrp}{\ncat{Grp}} % Category of groups
\newcommand{\catAb}{\ncat{Ab}} % Category of abelian groups
\newcommand{\catRing}{\ncat{Ring}} % Category of rings
\newcommand{\catFld}{\ncat{Fld}} % Category of fields

\newcommand{\catMod}[1]{{#1\text{-}\scat{Mod}}}
\newcommand{\catRMod}{\catMod{R}}

\newcommand{\End}{\mathrm{End}}
\newcommand{\Hom}{\mathrm{Hom}}

\DeclareMathOperator{\chr}{char}


%% Framed exercise environment

\mdfdefinestyle{swannexercise}{%
    skipabove=0.5em plus 0.4em minus 0.2em,
	skipbelow=0.5em plus 0.4em minus 0.2em,
	leftmargin=-5pt,
	rightmargin=-5pt,
	innerleftmargin=5pt,
	innerrightmargin=5pt,
	innertopmargin=5pt,
	innerbottommargin=4pt,
	linewidth=0pt,
	splittopskip=1.2em minus 0.2em,
	splitbottomskip=0.5em plus 0.2em minus 0.1em,
	backgroundcolor=backgroundcolor,
	frametitlebackgroundcolor=titlecolor,
	frametitlefont={\scshape},
    theoremseparator={},
    % theoremspace={},
	frametitleaboveskip=3pt,
	frametitlebelowskip=2pt
}

\mdtheorem[style=swannexercise]{exerciseframed}{Exercise}

\let\oldexerciseframed\exerciseframed
\renewcommand{\exerciseframed}{%
  \crefalias{theorem}{exerciseframed}%
  \oldexerciseframed}

\usepackage{listofitems}

\settocdepth{subsection}
\renewenvironment{exerciseframed}[1][]{%
    \setsepchar{.}%
    \readlist*\mylist{#1}%
    \def\smalllabel{\mylist[2].\mylist[3]}%
    \refstepcounter{exerciseframed}%
    % \addcontentsline{toc}{subsection}{Exercise \smalllabel}%
    \begin{exerciseframed*}[#1]%
    \label{ex:#1}%
}{%
    \end{exerciseframed*}%
}

% https://tex.stackexchange.com/a/23491/63353
\newcommand{\RNum}[1]{\uppercase\expandafter{\romannumeral #1\relax}}

\newcommand{\exref}[1]{%
    % \setsepchar{.}%
    % \readlist*\mylist{#1}%
    % \ifnum \arabic{chapter}=\mylist[1]
    %     \def\mylabel{\mylist[2].\mylist[3]}%
    % \else
    %     \def\mylabel{\RNum{\mylist[1]}.\mylist[2].\mylist[3]}%
    % \fi
    \hyperref[ex:#1]{Exercise~#1}%
}

\theoremstyle{nonumberplain}
\theoremsymbol{\ensuremath{\square}}
\newtheorem{solution}{Solution}

\let\oldsolution\solution
\renewcommand{\solution}{%
  \crefalias{theorem}{solution}%
  \oldsolution}

\newcommand{\solutionlabelfont}[1]{{\normalfont\color{linkcolor}#1}}
\newlist{solutionsec}{enumerate}{1}
\setlist[solutionsec]{leftmargin=0pt, parsep=0pt, listparindent=\parindent, font=\solutionlabelfont, label=(\alph*), labelsep=0pt, labelwidth=20pt, itemindent=20pt, align=left, itemsep=10pt}


% \renewcommand{\thechapter}{\Roman{chapter}}
% \renewcommand{\thesection}{\arabic{section}}

\DeclarePairedDelimiter{\ord}{\lvert}{\rvert}
\DeclareMathOperator{\lcm}{lcm}
\DeclareMathOperator{\Aut}{Aut}
\DeclareMathOperator{\Inn}{Inn}

\usepackage{caption} % Links to figures jump correctly
\Crefname{figure}{Figure}{Figures}


\newenvironment{displaytheorem}{%
	\begin{displayquote}\itshape%
}{%
	\end{displayquote}%
}


\newcommand{\upset}{\operatorname{\uparrow}}
\newcommand{\downset}{\operatorname{\downarrow}}
\newcommand{\matgroup}[3]{\mathrm{#1}_{#2}(#3)}
\newcommand{\GL}[2]{\matgroup{GL}{#1}{#2}}
\newcommand{\SL}[2]{\matgroup{SL}{#1}{#2}}
\newcommand{\catGSet}[1][G]{{#1\text{-}\catSet}}
\newcommand{\frakI}{\mathfrak{I}}
\newcommand{\field}{\mathbb{F}}
\let\bigcoprod\coprod
\renewcommand{\coprod}{\sqcup}


\begin{document}

\maketitle


\chapter{Basic theory of nets}

If $P$ is a preordered set, then a subset $A \subseteq P$ is said to be \emph{cofinal} in $P$ if for every $x \in P$ there is a $y \in A$ with $x \leq y$. A function $f \colon P \to Q$ between ordered sets is said to be \emph{cofinal} if $f(P)$ is cofinal in $Q$.


\newcommand{\mynet}{u}
\newcommand{\subnet}{v}
\newcommand{\altnet}{v}
\newcommand{\newnet}{w}
\newcommand{\mylimit}{x}
\newcommand{\calN}{\mathcal{N}}
\newcommand{\nhoods}[1]{\calN_{#1}}
\newcommand\tail[2][]{%
    \ifstrempty{#1}{%
        T_{#2}
    }{%
        T^{#1}_{#2}
    }%
}
\newcommand{\tails}[1]{\calT_{#1}}

\begin{definition}[Nets]
    Let $X$ be a set. A \emph{net} in $X$ is a function $\mynet \colon I \to X$, where $I$ is a nonempty directed set. The value $\mynet(i)$ at $i \in I$ is usually denoted $\mynet_i$.
    
    Given $i \in I$, the set
    %
    \begin{equation*}
        \tail{i}
            = \tail[\mynet]{i}
            = \set{\mynet_j \in A}{j \geq i}
    \end{equation*}
    %
    is called the \emph{tail of $\mynet$ following $i$}. The collection of tails of $\mynet$ is denoted $\tails{\mynet}$.
\end{definition}
%
Some authors allow the domain $I$ to be empty (e.g. \textcite{willard}, \cite{folland2007}), while others do not (e.g. \cite{kelley1975}, \cite{beardon1997}, who both require even directed sets to be nonempty). We require $I$ to be nonempty since it makes the correspondence between nets and filters nicer, since filters are required to be nonempty.

Notice that $\tails{\mynet}$ is downward directed with respect to set inclusion: For $i, j \in I$ there is a $k \in A$ with $i, j \leq k$, and hence $\tail{k} \subseteq \tail{i} \intersect \tail{j}$. Intuitively speaking, every pair of tails of $\mynet$ \enquote{meet} somewhere in the future.

\begin{definition}
    Let $\mynet \colon I \to X$ be a net, and let $A \subseteq X$.
    %
    \begin{enumdef}
        \item The net $\mynet$ is \emph{eventually in $A$} if there exists a tail $T \in \tails{\mynet}$ such that $T \subseteq A$. Equivalently, there exists an $i \in I$ such that $\mynet_j \in A$ for all $j \geq i$.

        \item The net $\mynet$ is \emph{frequently/cofinally in $A$} if $T \intersect A \neq \emptyset$ for all $T \in \tails{\mynet}$. Equivalently, for all $i \in I$ there exists a $j \geq i$ such that $\mynet_j \in E$, i.e. $\mynet\preim(A)$ is cofinal in $I$.
    \end{enumdef}
\end{definition}
%
In particular, since $\tails{\mynet}$ is downward directed, if $\mynet$ is eventually in $A$ then it is frequently in $A$. Notice also that $\mynet$ is \emph{not} eventually in $A$ if and only if it is frequently in $X \setminus A$.

\begin{definition}
    Let $\mynet \colon I \to X$ be a net in a topological space $X$, and let $\mylimit \in X$.
    %
    \begin{enumdef}
        \item The net $\mynet$ \emph{converges to $\mylimit$} if $\mynet$ is eventually in $N$ for all $N \in \nhoods{\mylimit}$, i.e. if
        %
        \begin{equation*}
            \forall N \in \nhoods{\mylimit} \; \exists T \in \tails{\mynet} \colon T \subseteq N.
        \end{equation*}
        %
        In this case we use either of the following notations:
        %
        \begin{equation*}
            \mynet \to \mylimit, \quad
            \mynet_i \to \mylimit, \quad
            \lim \mynet = \mylimit, \quad
            \lim_{i \in I} \mynet_i = \mylimit.
        \end{equation*}

        \item The point $\mylimit$ is a \emph{cluster point} of $\mynet$ if $\mynet$ is frequently in $N$ for all $N \in \nhoods{\mylimit}$, i.e. if
        %
        \begin{equation*}
            \forall N \in \nhoods{\mylimit}, T \in \tails{\mynet} \colon T \intersect N \neq \emptyset.
        \end{equation*}
    \end{enumdef}
\end{definition}
%
Thus if $\mynet \to \mylimit$, then $\mylimit$ is a cluster point of $\mynet$.

Next we consider \emph{subnets}. There are at least three different, non-equivalent definitions of subnets:

\begin{definition}[Subnets]
    \label{def:subnets}
    Let $\mynet \colon I \to X$ and $\subnet \colon J \to X$ be nets in a set $X$.
    
    \begin{enumdef}
        \item \label{enum:AA-subnet} We say that $\subnet$ is a \emph{(Aarnes--Andenæs) subnet} of $\mynet$ if for all $A \subseteq X$,
        %
        \begin{equation*}
            \text{$\mynet$ is eventually in $A$}
            \quad \implies \quad
            \text{$\subnet$ is eventually in $A$},
        \end{equation*}
        %
        or equivalently if
        %
        \begin{equation*}
            \text{$\subnet$ is frequently in $A$}
            \quad \implies \quad
            \text{$\mynet$ is frequently in $A$}.
        \end{equation*}

        \item \label{enum:Kelley-subnet} We say that $\subnet$ is a \emph{Kelley subnet} of $\mynet$ if there exists a function $\phi \colon J \to I$ such that
        %
        \begin{enumerate}
            \item $\subnet = \mynet \circ \phi$, and
            \item for each $i_0 \in I$ there is a $j_0 \in J$ such that $j \geq j_0$ implies that $\phi(j) \geq i_0$.
        \end{enumerate}

        \item \label{enum:Willard-subnet} We say that $\subnet$ is a \emph{Willard subnet} of $\mynet$ if there exists a function $\phi \colon J \to I$ such that
        %
        \begin{enumerate}
            \item $\subnet = \mynet \circ \phi$,
            \item $\phi$ is monotone, and
            \item $\phi$ is cofinal, i.e. for each $i_0 \in I$ there is a $j_0 \in J$ such that $\phi(j_0) \geq i_0$.
        \end{enumerate}
    \end{enumdef}
\end{definition}
%
It will turn out that Aarnes--Andenæs subnets (henceforth simply \enquote{AA subnets}) will be most important to us. We may equip the class of nets in a set $X$ with a preorder by letting $\subnet \leq \mynet$ if $\subnet$ is an AA subnet of $\mynet$. If both $\subnet \leq \mynet$ and $\mynet \leq \subnet$, then we say that $\mynet$ and $\subnet$ are \emph{equivalent} and write $\mynet \sim \subnet$. As with sequences, if $\mynet \to \mylimit$ and $\subnet$ is a subnet (of any kind) of $\mynet$, then also $\subnet \to \mylimit$. This is clear for AA subnets, and will follow for the other two kinds from the following remark:

\begin{remarkbreak}[Relationship between definitions of subnets]
    We claim that each Willard subnet is a Kelley subnet, and each Kelley subnet is an AA subnet. For the first claim, let $i_0 \in I$ and choose $j_0$ in accordance with \cref{enum:Willard-subnet}. For $j \in J$ with $j \geq j_0$ we thus have
    %
    \begin{equation*}
        \phi(j)
            \geq \phi(j_0)
            \geq i_0
    \end{equation*}
    %
    as desired.

    Next assume that $\subnet$ is a Kelley subnet of $\mynet$, assume that $\subnet$ is frequently in some $A \subseteq X$, and let $i_0 \in I$. Choose $j_0 \in J$ such that $j \geq j_0$ implies $\phi(j) \geq i_0$. There is some $j \in J$ such that $\mynet_{\phi(j)} = \subnet_j \in A$. Hence $\mynet$ is frequently in $A$, so $\subnet$ is an AA subnet.

    The converses do not hold. For counterexamples see \textcite[§7.16]{schechter1997} % TODO add counterexamples?
\end{remarkbreak}

\begin{lemma}
    \label{thm:Willard-subnet}
    Let $\mynet \colon I \to X$ and $\altnet \colon J \to X$ be nets in a set $X$ with the property that $T_1 \intersect T_2 \neq \emptyset$ for all $T_1 \in \tails{\mynet}$ and $T_2 \in \tails{\altnet}$. Then $\mynet$ and $\altnet$ have a common Willard subnet $\newnet$.

    Furthermore, $\newnet$ can be chosen to be maximal among common AA subnets of $\mynet$ and $\altnet$: That is, any common AA subnet of $\mynet$ and $\altnet$ is also an AA subnet of $\newnet$.
\end{lemma}
%
We have stated the lemma in terms of two nets, but the proof generalises in an obvious way to any finite number of nets. We shall not use this lemma nor the following corollary in the sequel, so we allow ourselves to use the theory of filters in the proof below.

\newcommand{\evfilt}[1][]{\calE_{#1}}
\newcommand{\evfilteq}[1][]{\tilde{\calE}_{#1}}

\begin{proof}
    For $i_0 \in I$ and $j_0 \in J$, notice that
    %
    \begin{equation*}
        \set{x \in X}{x = \mynet_i = \altnet_j \text{ for some } i \geq i_0, j \geq j_0}
            = \tail[\mynet]{i_0} \intersect \tail[\altnet]{j_0}
            \neq \emptyset.
    \end{equation*}
    %
    Hence the set
    %
    \begin{equation*}
        K
            = \set{(i,j) \in I \prod J}{\mynet_i = \altnet_j}
    \end{equation*}
    %
    is nonempty, and if $I \prod J$ is equipped with the product order, it is easy to see that $K$ is directed. Define a net $\newnet \colon K \to X$ by defining $\newnet_{(i,j)}$ as the common value $\mynet_i = \altnet_j$. Then $\newnet = \mynet \circ \pi_1|_K$, where $\pi_1 \colon I \prod J \to I$ is the projection onto $I$. This trivially satisfies the conditions in \cref{enum:Willard-subnet}, showing that $\newnet$ is a Willard subnet of both $\mynet$ and $\altnet$.
    
    It is also easy to see that
    %
    \begin{equation*}
        \tail[\newnet]{(i,j)}
            = \tail[\mynet]{i} \intersect \tail[\altnet]{j}
    \end{equation*}
    %
    for all $(i,j) \in K$. Hence
    %
    \begin{equation*}
        \evfilt[w]
            = \set{T_1 \intersect T_2}{T_1 \in \tails{u}, T_2 \in \tails{v}}
            = \set{F_1 \intersect F_2}{F_1 \in \evfilt[u], F_2 \in \evfilt[v]},
    \end{equation*}
    %
    so $\evfilt[w]$ is the minimal common superfilter of $\evfilt[u]$ and $\evfilt[v]$ It follows by \cref{thm:subnet-superfilter-correspondence} that $\newnet$ is maximal as claimed.
\end{proof}


\begin{corollary}
    Let $\mynet$ be a net, and let $\subnet$ be an AA subnet. Then $\mynet$ has a Willard subnet that is equivalent to $\subnet$.
\end{corollary}

\begin{proof}
    Notice that $\mynet$ and $\subnet$ have a common AA subnet, namely $\subnet$. But \cref{thm:Willard-subnet} yields a maximal common Willard subnet $\newnet$, and by maximality $\subnet$ is also a subnet of $\newnet$. Hence these are equivalent as claimed.
\end{proof}
%
Thus the three types of subnets can be used interchangably, insofar as the properties of a subnet are invariant up to equivalence. We will use AA subnets since their correspondence with superfilters is nicer, and henceforth \enquote{subnet} will mean AA subnet.



\chapter{Basic theory of filters}

\begin{definition}[Filters]
    Let $X$ be a set. A \emph{filter on $X$} is a proper filter $\calF$ in the powerset $\powerset{X}$ ordered by inclusion. That is, $\calF$ is a nonempty collection of subsets of $X$ that is
    %
    \begin{enumdef}
        \item \label{enum:filter-def-proper} proper, i.e. $\emptyset \not\in \calF$,
        \item \label{enum:filter-def-directed} downward directed, i.e., for $F_1, F_2 \in \calF$ there is an $F_3 \in \calF$ such that $F_3 \subseteq F_1, F_2$, and
        \item \label{enum:filter-def-closed} upward closed, i.e. $\calF = \calF^\uparrow$.
    \end{enumdef}
\end{definition}
%
The condition \subcref{enum:filter-def-closed} means that if $F \in F$ and $F \subseteq G$, then $G \in \calF$. In the presence of \subcref{enum:filter-def-closed}, condition \subcref{enum:filter-def-directed} is equivalent to $\calF$ being closed under (binary) intersections.

We want a way to generate filters from less restrictive collections of sets. \textcite[Exercise~2.22]{daveypriestley2002} gives a general way to do this, and we notice that if $\emptyset \neq \calB \subseteq \powerset{X}$ is already downward directed, then the filter generated by $\calB$ is just $\calB^\uparrow$. In fact, it is trivial to show that (in a general lattice) $\calB$ is downward directed if and only if $\calB^\uparrow$ is, so $\calB^\uparrow$ is a (not necessarily proper) filter if and only if $\calB$ is downward directed. If we further require that $\calB$ not contain the empty set, then $\calB^\uparrow$ is a filter in the above sense. This motivates the following definition:

\begin{definition}[Filter bases]
    Let $X$ be a set. A \emph{filter basis on $X$} is a nonempty collection $\calB$ of subsets of $X$ that is
    %
    \begin{enumdef}
        \item proper, and
        \item downward directed.
    \end{enumdef}
    %
    The filter \emph{generated by $\calB$} is the filter $\calB^\uparrow$. If $\calF$ is a filter on $X$ and $\calF = \calB^\uparrow$, then $\calB$ is called a \emph{basis} for $\calF$.
\end{definition}

If $X$ is a topological space and $\mylimit \in X$, then we denote the family of neighbourhoods of $\mylimit$ by $\nhoods{\mylimit}$. Notice that this is a filter on $X$, so we call it the \emph{neighbourhood filter} of $x$.

\begin{definition}
    Let $\calF$ be a filter on a topological space $X$, and let $\mylimit \in X$.
    %
    \begin{enumdef}
        \item The filter $\calF$ \emph{converges to $\mylimit$} if $\nhoods{\mylimit} \subseteq \calF$. In this case we write $\calF \to \mylimit$.
        
        \item \label{enum:filter-cluster} The point $\mylimit$ is called a \emph{cluster point} of $\calF$ if
        %
        \begin{equation*}
            \forall N \in \nhoods{\mylimit}, F \in \calF \colon F \intersect N \neq \emptyset.
        \end{equation*}
    \end{enumdef}
\end{definition}
%
Thus if $\calF \to \mylimit$, then $\mylimit$ is a cluster point of $\calF$. Notice the similarity between the definition of cluster points for nets and filters respectively.

\begin{remarkbreak}
    \begin{enumrem}
        \item \label{enum:filter-cluster-points} Notice also that if $\calB$ is a basis for the filter $\calF$ and $\nhoods{\mylimit} \subseteq \calB$, then $\calF \to \mylimit$. Furthermore, for every $F \in \calF$ there is a $B \in \calB$ with $B \subseteq F$. So if $B \intersect N \neq \emptyset$ then also $F \intersect N \neq \emptyset$. Hence we may also replace $\calF$ with $\calB$ in \subcref{enum:filter-cluster}.
        
        \item \label{enum:filter-cluster-equivalent} The set of cluster points of $\calF$ is given by
        %
        \begin{equation*}
            \bigintersect_{F \in \calF} \closure{F},
        \end{equation*}
        %
        since $x$ is a cluster point of $\calF$ if and only if every neighbourhood of $x$ intersects every $F \in \calF$.
    \end{enumrem}
\end{remarkbreak}

In the theory of filters, the concept corresponding to subnets is that of \emph{superfilters}:

\begin{definition}[Superfilters]
    Let $\calF$ and $\calG$ be filters on the same set $X$. We say that $\calG$ is a \emph{superfilter} of $\calF$ if $\calF \subseteq \calG$.
\end{definition}
%
We sometimes also say that $\calF$ is a \emph{subfilter} of $\calG$, or that $\calF$ is \emph{finer} than $\calG$ and $\calG$ \emph{courser} than $\calF$. The class of filters on a fixed set is of course partially ordered by inclusion.

Notice that if $\calF \to \mylimit$ and $\calG$ is a superfilter of $\calF$, then also $\calG \to \mylimit$ as we would expect if superfilters are to play the role of subnets.


\chapter{Correspondence between nets and filters}

\newcommand{\filter}{\mathfrak{F}}

\newcommand{\filters}[1]{\mathfrak{F}(#1)}
\newcommand{\nets}[1]{\mathfrak{N}(#1)}
\newcommand{\netseq}[1]{\tilde{\mathfrak{N}}(#1)}

\newcommand{\subord}{\vdash}

We wish to study the correspondence between nets and filters on a fixed set $X$. Given a net $\mynet$ in $X$, recall that the collection $\tails{\mynet}$ of tails of $\mynet$ is downward directed, so $(\tails{\mynet})^\uparrow$ is indeed a filter, namely the smallest filter containing all tails of $\mynet$. We call this the \emph{eventuality filter} of $\mynet$ and denote it by $\evfilt[\mynet]$. This defines a map $\evfilt \colon \nets{X} \to \filters{X}$ given by $\evfilt(\mynet) = \evfilt[\mynet]$.

Since superfilters correspond to subnets, some authors define an ordering $\subord$ on the set of filters by letting $\calG \subord \calF$ if $\calF \subseteq \calG$ (i.e. $\subord$ is dual to set inclusion). In this case we say that $\calG$ is \emph{subordinate} to $\calF$. We will not use this language in the sequel, so $\filters{X}$ is ordered by inclusion.

\begin{lemma}
    Let $\calF$ be a filter on a set $X$. There exists a net $\mynet$ in $X$ such that $\calF = \tails{\mynet}$. In particular $\calF = \evfilt[\mynet]$, so the map $\evfilt$ is surjective.
\end{lemma}

\begin{proof}
    Let $\calF$ be a filter on $X$ and define a direction on the set
    %
    \begin{equation*}
        I
            = \set{(x,F)}{x \in F \in \calF}
    \end{equation*}
    %
    by letting $(x,F) \leq (y,G)$ if $G \subseteq F$. Define a net $\mynet \colon I \to X$ by $\mynet_{(x,F)} = x$. Notice that
    %
    \begin{equation*}
        \tail{(x,F)}
            = \set{ \mynet_{(y,G)} }{ y \in G \subseteq F }
            = \set{ y }{ y \in G \subseteq F }
            = F,
    \end{equation*}
    %
    so each tail lies in $\calF$, and every element in $\calF$ is a tail. Hence $\calF = \tails{\mynet}$, which implies that
    %
    \begin{equation*}
        \evfilt[\mynet]
            = (\tails{\mynet})^\uparrow
            = \calF^\uparrow
            = \calF,
    \end{equation*}
    %
    since $\calF$ is already a filter.
\end{proof}

\begin{theorem}
    \label{thm:subnet-superfilter-correspondence}
    Let $\mynet$ and $\subnet$ be nets in a set $X$. For $F \subseteq X$ we have
    %
    \begin{enumthm}
        \item \label{enum:net-filter-eventuality} $F \in \evfilt[\mynet]$ if and only if $\mynet$ is eventually in $F$, and
        \item \label{enum:net-filter-ordering} $\subnet \leq \mynet$ if and only if $\evfilt[\mynet] \subseteq \evfilt[\subnet]$.
    \end{enumthm}
    %
    In particular, the map $\evfilt \colon (\nets{X}, \leq) \to (\filters{X}, \subseteq)$ is decreasing.
\end{theorem}

\begin{proof}
\begin{proofsec}
    \item[\subcref{enum:net-filter-eventuality}]
    We have $F \in \evfilt[\mynet] = (\tails{\mynet})^\uparrow$ if and only if $F$ contains a tail $T$ of $\mynet$, and this is the case if and only if $\mynet$ is eventually in $F$.

    \item[\subcref{enum:net-filter-ordering}]
    The filter $\evfilt[\mynet]$ contains those subsets $F \subseteq X$ such that $\mynet$ is eventually in $F$, so this follows immediately from \subcref{enum:net-filter-eventuality}.
\end{proofsec}
\end{proof}

\newcommand{\myinv}[1][]{\tilde{\mathfrak{n}}_{#1}}

This motivates the following construction: If $\mynet$ is a net in $X$, denote by $[\mynet]$ the $\sim$-equivalence class of $\mynet$. \cref{thm:subnet-superfilter-correspondence} then says that the map $\evfilt$ induces a map $\evfilteq \colon \netseq{X} \to \filters{X}$ which sends a class $[\mynet]$ to $\evfilt[\mynet]$, and that this map is injective. It inherits surjectivity from $\evfilt$, so it is in fact a bijection, hence an order antiisomorphism. We denote its inverse by $\myinv$ and write $\myinv(\calF) = \myinv[\calF]$.


\begin{corollary}
    Let $\mynet$ be a net in a topological space $X$, and let $\mylimit \in X$. We then have that
    %
    \begin{enumcor}
        \item \label{enum:net-filter-convergence} $\mynet \to \mylimit$ if and only if $\evfilt[\mynet] \to \mylimit$, and
        \item \label{enum:net-filter-cluster} $\mylimit$ is a cluster point of $\mynet$ iff it is a cluster point of $\evfilt[\mynet]$.
    \end{enumcor}
\end{corollary}

\begin{proof}
    \begin{proofsec}
        \item[\subcref{enum:net-filter-convergence}]
        Apply \cref{enum:net-filter-eventuality} to each neighbourhood in $\nhoods{\mylimit}$.

        \item[\subcref{enum:net-filter-cluster}]
        Notice that $\mylimit$ is a cluster point of $\mynet$ if and only if
        %
        \begin{equation*}
            \forall N \in \nhoods{\mylimit}, T \in \tails{\mynet} \colon N \intersect T \neq \emptyset.
        \end{equation*}
        %
        Since $\tails{\mynet}$ is a basis for $\evfilt[\mynet]$, by \cref{enum:filter-cluster-points} the above holds if and only if $\mylimit$ is a cluster point of $\evfilt[\mynet]$.
    \end{proofsec}
\end{proof}


\chapter{Cluster points}

Next we see that cluster points can be characterised in terms of subnets and superfilters, just as limit points can in terms of sequences in a metric space.

\begin{proposition}
    Let $X$ be a topological space, $\mynet$ a net in and $\calF$ a filter on $X$. For $\mylimit \in X$ we have that
    \begin{enumprop}
        \item \label{enum:superfilter-cluster} $\mylimit$ is a cluster point of $\calF$ iff $\calF$ has a superfilter converging to $\mylimit$,
    \end{enumprop}
    %
    and equivalently that
    %
    \begin{enumprop}[resume]
        \item \label{enum:subnet-cluster} $\mylimit$ is a cluster point of $\mynet$ iff $\mynet$ has a subnet converging to $\mylimit$.
    \end{enumprop}
\end{proposition}

\begin{proof}
\begin{proofsec}
    \item[\subcref{enum:superfilter-cluster}]
    First assume that $\mylimit$ is a cluster point of $\calF$ and consider the collection of sets
    %
    \begin{equation*}
        \calB
            = \set{F \intersect N}{F \in \calF, N \in \nhoods{\mylimit}}
            \neq \emptyset.
    \end{equation*}
    %
    Notice that every element in $\calB$ is nonempty, and that
    %
    \begin{equation*}
        (F_1 \intersect N_1) \intersect (F_2 \intersect N_2)
            = (F_1 \intersect F_2) \intersect (N_1 \intersect N_2)
            \in \calB,
    \end{equation*}
    %
    so $\calB$ is a filter basis. Denote the filter it generates by $\calG$. This is a superfilter of $\calF$ since $X \in \nhoods{\mylimit}$, and we have $\nhoods{\mylimit} \subseteq \calG$ since $X \in \calF$, so $\calG \to \mylimit$.

    Conversely assume that $\calF$ has a superfilter $\calG$ converging to $\mylimit$. This means that $\nhoods{\mylimit} \subseteq \calG$, so for all $F \in \calF$ and $N \in \nhoods{\mylimit}$ the intersection $F \intersect N$ lies in $\calG$ since $\calG$ is a filter, hence is nonempty.

    \item[\subcref{enum:subnet-cluster}]
    Recall that $\mylimit$ is a cluster point $\mynet$ if and only if it is a cluster point of $\evfilt[\mynet]$, which by \subcref{enum:superfilter-cluster} is the case just when $\evfilt[\mynet]$ has a superfilter converging to $\mylimit$. But such a superfilter corresponds to a subnet of $\mynet$, and this converges if and only the superfilter does.
\end{proofsec}
\end{proof}


\chapter{Ultrafilters and ultranets}

\begin{definition}[Ultrafilters]
    A filter $\calU$ on a set $X$ is said to be an \emph{ultrafilter} if it is a maximal element in $\filters{X}$, i.e. if $\calU \subseteq \calF$ implies that $\calU = \calF$ for any filter $\calF$ on $X$.
\end{definition}
%
We have the following characterisation of ultrafilters which motivates the definition of ultranets below, and which we will use in the proof of \cref{thm:ultrafilter-pushforward}.

\begin{proposition}
    \label{thm:ultrafilter-equivalent-condition}
    A filter $\calU$ on a set $X$ is an ultrafilter if and only if for every $A \subseteq X$ either $A \in \calU$ or $X \setminus A \in \calU$.
\end{proposition}

\begin{proof}
    Assume that there is an $A \subseteq X$ such that neither $A$ nor $X \setminus A$ is an element of $\calU$. Notice that then $A \intersect U \neq \emptyset$ for all $U \in \calU$, since otherwise $U \subseteq X \setminus A$, and so we would have $X \setminus A \in \calU$. The set
    %
    \begin{equation*}
        \calB
            = \set{A \intersect U}{U \in \calU}
    \end{equation*}
    %
    is then easily seen to be a filter basis. Let $\calF$ be the filter generated by $\calB$. Clearly $\calU \subseteq \calF$ since $A \intersect U \subseteq U$ for any $U \in \calU$. But since $X \in \calU$ we also have $A \in \calF$, so $\calF$ is strictly larger than $\calU$. Hence $\calU$ is not an ultrafilter.

    Conversely let $\calF$ be a not necessarily proper filter on $X$ that is strictly greater than $\calU$, and let $A \in \calF \setminus \calU$. Then we must have $X \setminus A \in \calU$, so $\emptyset = A \intersect X \setminus A \in \calF$, and hence $\calF = \powerset{X}$.
\end{proof}

The corresponding notion for nets is usually defined as follows:

\begin{definition}
    A net $\mynet$ in a set $X$ is an \emph{ultranet} or \emph{universal net} if for every $A \subseteq X$, $\mynet$ is eventually in either $A$ or $X \setminus A$.
\end{definition}
%
\cref{thm:ultrafilter-equivalent-condition} thus shows that $\mynet$ is an ultranet if and only if $\evfilt[\mynet]$ is an ultrafilter.


\begin{theorem}
    \label{thm:ultrafilter-existence}
    Every filter is contained in an ultrafilter. Equivalently, every net has a universal subnet.
\end{theorem}

\begin{proof}
    Let $\calF$ be a filter on a set $X$, and consider the set
    %
    \begin{equation*}
        \mathbb{F}
            = \set{\calG \in \filters{X}}{\calF \subseteq \calG}.
    \end{equation*}
    %
    This is a nonempty partially ordered set. If $\mathbb{G}$ is a chain in $\mathbb{F}$, then it is easy to see that $\bigunion_{\calG \in \mathbb{G}} \calG \in \mathbb{F}$. Hence every chain in $\mathbb{F}$ is bounded, so Zorn's lemma yields a maximal element $\calU$. Clearly $\calU$ is an ultrafilter containing $\calF$ as desired.
\end{proof}


\chapter{Continuity}

\begin{definition}[Pushforward of nets and filters]
    Let $f \colon X \to Y$ be a function, $\mynet \colon I \to X$ a net and $\calF$ a filter on $X$.
    %
    \begin{enumdef}
        \item The \emph{pushforward of $\mynet$ by $f$} is the net $f(\mynet) = f \circ \mynet \colon I \to Y$.
        \item The \emph{pushforward of $\calF$ by $f$} is the filter $f(\calF)$ generated by the filter basis $\set{f(F)}{F \in \calF}$.
    \end{enumdef}
\end{definition}

\begin{remarkbreak}
    \begin{enumrem}
        \item If $\mynet \sim \altnet$ then also $f(\mynet) \sim f(\altnet)$. We therefore define the pushforward $f([\mynet])$ of the class $[\mynet]$ by the class $[f(\mynet)]$.
        
        \item If $\calB$ is a basis for the filter $\calF$, then consider the filter $f(\calB)$ generated by the filter basis $\set{f(B)}{B \in \calB}$. We claim that $f(\calB) = f(\calF)$. The inclusion \enquote{$\subseteq$} is clear, and if $G = f(F)$ for some $F \in \calF$, then $B \subseteq F$ for some $B \in \calB$. Hence $f(B) \subseteq G$, so $G \in f(\calB)$ since the latter is upward closed.
        
        \item \label{enum:pushforward-filter-equivalent} We may give a different characterisation of the pushforward $f(\calF)$. Consider the set
        %
        \begin{equation*}
            \calG
                = \set{G \subseteq Y}{f\preim(G) \in \calF}.
        \end{equation*}
        %
        This is easily seen to be a filter. We claim that $f(\calF) = \calG$. For all $F \in \calF$ we have $F \subseteq f\preim(f(F))$, so $f\preim(f(F)) \in \calF$, and hence $f(F) \in \calG$. It follows that $f(\calF) \subseteq \calG$.

        Conversely, if $G \in \calG$ then $f(f\preim(G)) \subseteq G$. Since $f\preim(G) \in \calF$ we have $G \in f(\calF)$, so $\calG \subseteq f(\calF)$.
    \end{enumrem}
\end{remarkbreak}

\begin{lemma}
    Let $f \colon X \to Y$ be a function, and let $\mynet$ be a net in $X$. Then $f(\evfilt[\mynet]) = \evfilt[f(\mynet)]$. In particular $f(\evfilteq[{[u]}]) = \evfilteq[f({[u]})]$, so $\myinv[f(\calF)] = f(\myinv[\calF])$ for every filter $\calF$ on $X$.
\end{lemma}

\begin{proof}
    Since the tails $\tails{\mynet}$ is a basis for $\evfilt[\mynet]$, the collection $\set{f(T)}{T \in \tails{\mynet}}$ is a basis for $f(\evfilt[\mynet])$. On the other hand, this is precisely $\tails{f(\mynet)}$, which is a basis for $\evfilt[f(\mynet)]$.
\end{proof}


\begin{corollary}
    \label{thm:filter-net-convergence}
    Let $f \colon X \to Y$ be a function, $\mynet$ a net in $X$, and $y \in Y$. Then $f(\mynet) \to y$ if and only if $f(\evfilt[\mynet]) \to y$.
\end{corollary}

\begin{proof}
    This is immediate since $f(\evfilt[\mynet]) = \evfilt[f(\mynet)]$.
\end{proof}


\begin{proposition}
    \label{thm:continuity-filter-net}
    Let $f \colon X \to Y$ be a function between topological spaces, and let $x \in X$. Then $f$ is continuous at $x$ if and only if $\nhoods{f(x)} \subseteq f(\nhoods{x})$. In particular, the following are equivalent:
    %
    \begin{enumprop}
        \item \label{enum:continuity-point} $f$ is continuous at $x$.
        \item \label{enum:filter-convergence-point} For all filters $\calF$ on $X$, $\calF \to x$ implies that $f(\calF) \to f(x)$.
        \item \label{enum:net-convergence-point} For all nets $\mynet$ in $X$, $\mynet \to x$ implies that $f(\mynet) \to f(x)$.
    \end{enumprop}
\end{proposition}

\begin{proof}
    If $f$ is continuous at $x$ and $N \in \nhoods{f(x)}$, then there exists an $M \in \nhoods{x}$ such that $f(M) \subseteq N$. But $f(M)$ lies in $f(\nhoods{x})$, and hence so does $N$.

    Conversely, if $\nhoods{f(x)} \subseteq f(\nhoods{x})$ and $N \in \nhoods{f(x)}$, then by definition of $f(\nhoods{x})$ there exists an $M \in \nhoods{x}$ such that $f(M) \subseteq N$, so $f$ is continuous at $x$.
%
\begin{proofsec}
    % \item[\subcref{enum:continuity-point} $\implies$ \subcref{enum:net-convergence-point}]
    % For each $N \in \nhoods{f(x)}$ there is an $M \in \nhoods{x}$ such that $f(M) \subseteq N$. Since $\mynet$ is eventually in $M$, $f(\mynet) = f \circ \mynet$ is eventually in $f(M) \subseteq N$, showing that $f(\mynet) \to f(x)$.
    
    \item[\subcref{enum:continuity-point} $\implies$ \subcref{enum:filter-convergence-point}]
    Recall that $\calF \to x$ means that $\nhoods{x} \subseteq \calF$. Hence $\nhoods{f(x)} \subseteq f(\nhoods{x}) \subseteq f(\calF)$, so $f(\calF) \to f(x)$.

    \item[\subcref{enum:filter-convergence-point} $\implies$ \subcref{enum:continuity-point}]
    This follows by setting $\calF = \nhoods{x}$.

    \item[\subcref{enum:filter-convergence-point} $\Leftrightarrow$ \subcref{enum:net-convergence-point}]
    This follows immediately from \cref{thm:filter-net-convergence}
\end{proofsec}
\end{proof}


\begin{proposition}
    Let $\{X_\alpha\}_{\alpha \in A}$ be a family of topological spaces, and set $X = \bigprod_{\alpha \in A} X_\alpha$. Let $\mynet$ be a net in $X$, and let $\calF$ be a filter on $X$. Then we have for $x \in X$ that
    %
    \begin{enumprop}
        \item \label{enum:product-filter-convergence} $\calF \to x$ if and only if $\pi_\alpha(\calF) \to x_\alpha$ for all $\alpha \in A$, and that
        \item \label{enum:product-net-convergence} $\mynet \to x$ if and only if $\pi_\alpha(\mynet) \to x_\alpha$ for all $\alpha \in A$.
    \end{enumprop}
\end{proposition}

\begin{proof}
    We prove \subcref{enum:product-filter-convergence}, and \subcref{enum:product-net-convergence} is an immediate consequence. The \enquote{only if} part follows from \cref{thm:continuity-filter-net} since each $\pi_\alpha$ is continuous.
    
    Conversely let $N \in \nhoods{x}$, and let $U \in \nhoods{x_\alpha}$ be such that $\pi_\alpha\preim(U) \subseteq N$ for some $\alpha \in A$. Since $\calF$ is a filter, to show that $N \in \calF$ it suffices to show that $\pi_\alpha\preim(U) \in \calF$. Since $\pi_\alpha(\calF) \to x_\alpha$ we have $U \in \pi_\alpha(\calF)$, and by \cref{enum:pushforward-filter-equivalent} it follows that $\pi_\alpha\preim(U) \in \calF$. Hence $\calF \to x$ as desired.
\end{proof}


\begin{proposition}
    \label{thm:ultrafilter-pushforward}
    Let $f \colon X \to Y$ be a function, $\mynet$ an ultranet in $X$ and $\calU$ an ultrafilter on $X$. Then
    %
    \begin{enumprop}
        \item \label{enum:ultrafilter-pushforward} $f(\calU)$ is an ultrafilter on $Y$, and
        \item \label{enum:ultranet-pushforward} $f(\mynet)$ is an ultranet on $Y$.
    \end{enumprop}
\end{proposition}

\begin{proof}
    As usual, \subcref{enum:ultranet-pushforward} follows immediately from \subcref{enum:ultrafilter-pushforward}. To prove \subcref{enum:ultrafilter-pushforward} we use \cref{thm:ultrafilter-equivalent-condition}, so let $B \subseteq Y$. Then either $f\preim(B) \in \calU$ or $f\preim(Y \setminus B) = X \setminus f\preim(B) \in \calU$. By \cref{enum:pushforward-filter-equivalent} either $B \in f(\calU)$ or $Y \setminus B \in f(\calU)$, so $f(\calU)$ is an ultrafilter.
\end{proof}


\chapter{Compactness}

\newcommand{\calK}{\mathcal{K}}

\begin{proposition}
    \label{thm:compactness-equivalent}
    Let $X$ be a topological space. The following are equivalent:
    %
    \begin{enumprop}
        \item \label{enum:space-compact} $X$ is compact.
        \item \label{enum:every-filter-cluster} Every filter on $X$ has a cluster point.
        \item \label{enum:every-filter-convergent-superfilter} Every filter on $X$ has a convergent superfilter.
        \item \label{enum:every-ultrafilter-converges} Every ultrafilter on $X$ converges.
    \end{enumprop}
    %
    These are equivalent to the corresponding claims concerning nets:
    \begin{enumprop}[resume]
        \item Every net in $X$ has a cluster point.
        \item Every net in $X$ has a convergent subnet.
        \item Every ultranet in $X$ converges.
    \end{enumprop}
\end{proposition}
%
We will use the following well-known characterisation of compactness: The space $X$ is compact if and only if for every collection $\calA$ of closed subsets with the finite intersection property, $\bigintersect_{A \in \calA} A \neq \emptyset$.

\begin{proof}
\begin{proofsec}
    \item[\subcref{enum:space-compact} $\implies$ \subcref{enum:every-filter-cluster}]
    Let $\calF$ be a filter on $X$. Clearly $\calF$ itself has the finite intersection property, and so does the collection $\set{ \closure{F}}{F \in \calF}$. Hence $\bigintersect_{F \in \calF} \closure{F} \neq \emptyset$, so $\calF$ has a cluster point by \cref{enum:filter-cluster-equivalent}.

    \item[\subcref{enum:every-filter-cluster} $\implies$ \subcref{enum:every-filter-convergent-superfilter}]
    This is an immediate consequence of \cref{enum:superfilter-cluster}.

    \item[\subcref{enum:every-filter-convergent-superfilter} $\implies$ \subcref{enum:every-ultrafilter-converges}]
    If $\calU$ is an ultrafilter on $X$, then it has a convergent superfilter $\calF$. But since $\calU = \calF$, $\calU$ itself must be convergent.

    \item[\subcref{enum:every-ultrafilter-converges} $\implies$ \subcref{enum:space-compact}]
    Assume that there is a collection $\calA$ of closed subsets of $X$ with the finite intersection property and with $\bigintersect_{A \in \calA} A = \emptyset$. The collection
    %
    \begin{equation*}
        \calB
            = \set[\bigg]{ \bigintersect_{i=1}^n A_i }{ n \in \ints_+, A_1, \ldots, A_n \in \calA }
    \end{equation*}
    %
    is clearly a filter basis, so let $\calF$ be the filter it generates. By \cref{thm:ultrafilter-existence} $\calF$ is contained in an ultrafilter $\calU$. Now notice that
    %
    \begin{equation*}
        \bigintersect_{B \in \calU} \closure{B}
            \subseteq \bigintersect_{B \in \calB} \closure{B}
            = \bigintersect_{B \in \calB} B
            \subseteq \bigintersect_{A \in \calA} A
            = \emptyset,
    \end{equation*}
    %
    where the first equality follows since each $B \in \calB$ is closed. Thus $\calU$ has no cluster points, so it does not converge.
\end{proofsec}
\end{proof}


\begin{theorem}[Tychonoff's theorem]
    Let $\{X_\alpha\}_{\alpha \in A}$ be a collection of compact topological spaces. Then the product $\bigprod_{\alpha \in A} X_\alpha$ is also compact.
\end{theorem}

\begin{proof}
    Let $\calU$ be an ultrafilter in $X = \bigprod_{\alpha \in A} X_\alpha$. For all $\alpha \in A$, $\pi_\alpha(\calU)$ is an ultrafilter in $X_\alpha$ by \cref{enum:ultrafilter-pushforward}, so since $X_\alpha$ is compact \cref{thm:compactness-equivalent} implies that $\pi_\alpha(\calU)$ converges. \cref{enum:product-filter-convergence} in turn implies that $\calU$ converges, so another application of \cref{thm:compactness-equivalent} yields compactness of $X$.
\end{proof}


\nocite{*}

\printbibliography

\end{document}